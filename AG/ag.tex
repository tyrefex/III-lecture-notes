\documentclass[12pt]{article}

\usepackage{ishn}

\makeindex[intoc]

\begin{document}

\hypersetup{pageanchor=false}
\begin{titlepage}
	\begin{center}
		\vspace*{1em}
		\Huge
		\textbf{III Algebraic Geometry}

		\vspace{1em}
		\large
		Ishan Nath, Michaelmas 2024

		\vspace{1.5em}

		\Large

		Based on Lectures by Prof. Mark Gross

		\vspace{1em}

		\large
		\today
	\end{center}
	
\end{titlepage}
\hypersetup{pageanchor=true}

\tableofcontents

\newpage

% lecture 1

\setcounter{section}{-1}

\section{Introduction}%
\label{sec:intro}

Introductory reading by Hassett and Reid.

More commutative algebra by Atiyah and Macdonald, and Matsumura.

Standard AG texts by Hartshorne, G\"ortz-Wedhorn, and Ravi Vakil.

\subsection{Recap}%
\label{sub:recap}

In undergraduate, we fix an algebraically closed field $K$, and define affine $n$-space $\mathbb{A}^n = K^n$, and for an ideal $I \subseteq K[x_1, \ldots, x_n]$ we define
\[
	Z(I) = \{(a_1, \ldots, a_n) \in \mathbb{A}^n \mid f(a_1, \ldots, a_n) = 0 \text{ for all } f \in I \} \subseteq \mathbb{A}^n.
\]
We can define a topology on $\mathbb{A}^n$ by taking the closed sets to be the sets of the form $Z(I)$.

This turns out to be no good. We will instead introduce schemes. Natural questions are; why schemes, and why not varieties? Well,
\begin{itemize}
	\item With varieties, we always work with algebraically closed fields, to relate the algebra with the geometry. If $K = \mathbb{R}$, and $I = (x^2 + y^2 + 1) \subseteq K[x, y]$, then $Z(I) = \emptyset$, losing information about $I$.
	\item We may be interested in Diophantine equations, where the natural space is $\mathbb{Z}$.
	\item Even if $K$ is algebraically closed, we lose information passing from $I$ to $Z(I)$. For example if $I = (x^2)$, $Z = \{0\}$, then $I(Z(I)) = (x)$.

		But it is natural to consider ideals like $(x^2)$, for example considering $(y-x^2, y - \alpha) \subseteq \mathbb{C}[x, y]$. This produces two points for $\alpha \neq 0$, but only one point if $\alpha = 0$, but with some multiplicity.
\end{itemize}

\subsection{Categorical Philosophy}%
\label{sub:cat_phil}

Let $\mathbf{Set}$ be the category of sets. $\mathbf{Set}$ is the category with objects being all sets with morphisms between objects being maps of sets. If $X, Y$ are sets, we write $\mathrm{Hom}(X, Y)$ for the set of maps between $X$ and $Y$.

Note that there is a bijection $\mathrm{Hom}(\{\ast\}, X) \to X$ given by $(f : \{\ast\} \to X) \mapsto f(\ast)$.

We can use this philosophy to understand points on affine algebraic varieties. Nota $\mathbb{A}^0$ is a point. If $X$ is an affine variety, then the points of $X$ should be in one-to-one correspondence with $\mathrm{Hom}(\mathbb{A}^0, X)$.

Recall morphisms of affine varieties. Denote $A(X)$ by $K[x_1, \ldots, x_n]/I(X)$, where $I(X) = \{f \in K[x_1, \ldots, x_n] \mid f|_X = 0\}$. $A(X)$ is the \emph{coordinate ring} of $X$, a $K$-algebra.

We showed that if $X, Y$ are affine varieties, then
\[
\mathrm{Hom}(X, Y) = \mathrm{Hom}(A(Y), A(X)).
\]
So,
\[
	\mathrm{Hom}(\mathbb{A}^0, X) = \mathrm{Hom}(K[x_1, \ldots, x_n]/I(X), K).
\]
Note giving a $K$-algebra homomorphism $K[x_1, \ldots, x_n]/I(X) \to K$ can be done by specifying the images of $x_1$, say $x_1 \mapsto a_1$, such that, for any $f \in I(X)$, $f(a_1, \ldots, a_n) = 0$. So there is a one-to-one correspondence between such $K$-algebra homomorphisms, and points of $X$.

If $K$ is algebraically closed, the maximal ideals of $K[x_1, \ldots, x_n]$ are precisely ideals of the form $(x_1-a_1, \ldots, x_n-a_n)$ by Hilbert's Nullstellensatz. Similarly, for $A(X)$, the maximal ideals are $(x_1-a_1, \ldots, x_n-a_n)$ mod $I(X)$, with $(a_1, \ldots, a_n) \in X$.

Thus there is a bijection between points on $X$, and the maximal ideals of $A(X)$. This gives three objects, $X$, the homomorphisms and the maximal ideals, which are all bijective.

Now suppose $K$ is not algebraically closed. Consider the $K$-algebra homomorphisms $A(X) \to L$, where $L$ is an extension of $K$. If $x_i \mapsto a_i$, then $f(a_1, \ldots, a_n) = 0$ for all $f \in I(X)$. Thus,
\[
	\mathrm{Hom}_K(A(X), L) = \{(a_1, \ldots, a_n) \in \mathbb{A}^n \mid f(a_1, \ldots, a_n) = 0 \text{ for all } f \in I(X)\}.
\]

%lecture 2

These correspond to $L$-valued points.

We could also work over $\mathbb{Z}$. Take an ideal $I \subseteq \mathbb{Z}[x_1, \ldots, x_n]$, and $A = \mathbb{Z}[x_1, \ldots, x_n]/I$.

Then ring homomorphisms $A \to \mathbb{Z}$ are in one-to-one correspondence with points $(a_1, \ldots, a_n) \in \mathbb{Z}^n$ such that $f(a_1, \ldots, a_n) = 0$ for all $f \in I$.

Moreover maps $A \to \mathbb{F}_p$ give solutions mod $p$, and $A \to \mathbb{Q}$ give rational solutions.

\subsection{What we want}%
\label{sub:www}

Given a ring $A$, we want to define a gadget
\[
X = \Spec A,
\]
and an $R$-valued point of $X$ is a ring-homomorphism $A \to R$. We write the set of $R$-valued points as
\[
X(R) = \Hom(A, R).
\]
Morphisms $\Spec B \to \Spec A$ should be the same as ring homomorphisms $A \to B$. In category theory,

\begin{definition}
	The category of affine schemes it the \emph{opposite category} of rings.
\end{definition}

\textbf{Reminder:} In this course, all of our rings are unital, are commutative, and ring homomorphism $\phi : A \to B$ satisfy $\phi(1) = 1$.

\begin{definition}
	A \emph{scheme}\index{scheme} is a geometric object which is locally an affine scheme.
\end{definition}

Currently this is a nonsensical definition, which we will be trying to make sense of. The motivating example is the manifold, which locally looks like an open subset of $\mathbb{R}^n$.

\subsection{Introductory Definitions}%
\label{sub:id}

\begin{definition}\index{spectrum}
	Let $A$ be a ring. Then,
	\[
		\Spec A = \{p \subseteq A \mid p \text{ is a prime ideal}\}.
	\]
\end{definition}

In general, if we have an $L$-valued point of $X = Z(I) \subseteq \mathbb{A}^n$, we get a ring homomorphism $\phi : A(X) \to L$, which has image an integral subdomain of $L$, and so $\Ker \phi$ is prime.

\begin{definition}
	For $I \subseteq A$ an ideal, define
	\[
		V(I) = \{p \in \Spec A \mid p \supseteq I\}.
	\]
\end{definition}
Again recall $p$ is no longer a point, but a prime ideal.

\begin{proposition}
	The sets $V(I)$ form the closed sets of a topology on $\Spec A$, the \emph{Zariski topology}\index{Zariski topology}.
\end{proposition}

\begin{proofbox}
	Need to check a handful of things.
	\begin{itemize}
		\item $V(A) = \emptyset$, so $\emptyset$ is closed.
		\item $V(0) = \Spec A$, so $\Spec A$ is closed.
		\item If $\{I_j\}_{j \in J}$ is a collection of ideals, then note
			\[
			\bigcap_{j \in J} V(I_j)= V \left( \sum_{j \in J} I_j \right).
			\]
		\item We show that $V(I_1) \cup V(I_2) = V(I_1 \cap I_2)$. Indeed, if $p \supseteq I_1$ or $p \supseteq I_2$, then $p \supseteq I_1 \cap I_2$.

			In the other direction, then $p \supseteq I_1 \cap I_2$, then $p \supseteq I_1$ or $p \supseteq I_2$. This was proven in Part II. Or see Atiyah + Macdonald.

			This is easy: if $I_1, I_2 \not \subseteq p$, then there exists $i_1, i_2 \in I_1, I_2$ respectively that are not in $p$. But now $i_1 i_2 \in I_1 \cap I_2 \subseteq p$, so $i_1 i_2 \in p$.

			However $p$ is prime, so either $i_1 \in p$ or $i_2 \in p$, contradiction.
	\end{itemize}
\end{proofbox}

\begin{exbox}
	Consider $A = K[x_1, \ldots, x_n]$ with $K$ algebraically closed. For $I \subseteq A$, the maximal ideals of $A$ corresponding to points of $Z(I)$ are precisely the maximal ideals containing $I$.
\end{exbox}

\newpage

\section{Sheaves}%
\label{sec:sheaf}

Fix a topological space $X$.

\begin{definition}
	A \emph{presheaf}\index{presheaf} $\mathcal{F}$ on $X$ consists of data, such that:
	\begin{itemize}
			\item For every open set $U \subseteq X$, we have an abelian group $\mathcal{F}(U)$ (or more generally any element of a category).
			\item Whenever $V \subseteq U \subseteq X$ is open, there is a \emph{restriction}\index{restriction} homomorphism
				\[
				\rho_{UV} : \mathcal{F}(U) \to \mathcal{F}(V),
				\]
				such that $\rho_{UU} = \id_{\mathcal{F}(U)}$, and if $W \subseteq V \subseteq U \subseteq X$, then
				\[
				\rho_{UW} = \rho_{VW} \circ \rho_{UV}.
				\]
	\end{itemize}	
\end{definition}

\begin{remark}
	This is precisely a contravariant functor from the category of open sets to the category of abelian groups. As mentioned, we may replace the category of abelian groups with any category.
\end{remark}

\begin{definition}
	If $\mathcal{F}, \mathcal{G}$ are presheaves on $X$, then a \emph{morphism}\index{morphism of presheaves} $f : \mathcal{F} \to \mathcal{G}$ is data for each $U \subseteq X$, a group homomorphism $f_U : \mathcal{F}(U) \to \mathcal{G}(U)$ such that whenever $V \subseteq U$, we have a commutative diagram
	\[
	\begin{tikzcd}
		\mathcal{F}(U) \arrow[r, "f_U"] \arrow[d, "\rho_{UV}^{\mathcal{F}}"] & \mathcal{G}(U) \arrow[d, "\rho_{UV}^{\mathcal{G}}"] \\
		\mathcal{F}(V) \arrow[r, "f_V"] & \mathcal{G}(V)
	\end{tikzcd}
	\]
\end{definition}

\begin{definition}
	A presheaf $\mathcal{F}$ on $X$ is a \emph{sheaf}\index{sheaf} if it satisfies:
	\begin{enumerate}
		\item If $U \subseteq X$ is covered by $\{U_i\}$, and $s \in \mathcal{F}(U)$ such that $s|_{U_i} = \rho_{UU_i}(s) = 0$ for all $i$, then $s = 0$.
		\item If $U, \{U_i\}$ are as in the above, and $s_i \in \mathcal{F}(U_i)$ for each $i$ such that
			\[
			s_i |_{U_i \cap U_j} = s_j|_{U_i \cap U_j}
			\]
			for all $i, j$, then there exists $s \in \mathcal{F}(U)$ such that
			\[
			s|_{U_i} = s_i
			\]
			for all $i$.
	\end{enumerate}
	
\end{definition}

% lecture 3

\begin{remark}
	\begin{itemize}
		\item[]
		\item If $\mathcal{F}$ is a sheaf, then $\mathcal{F}(\emptyset) = 0$, since the empty cover is a cover of $\emptyset$.
		\item The two conditions together can be stated by saying
			\[
			\begin{tikzcd}
				0 \arrow[r, ""] & \mathcal{F}(U) \arrow[r, "\alpha"] & \bigoplus_{i \in I} \mathcal{F}(U_i) \arrow[r, "\beta_1", shift left = 2pt] \arrow[r, "\beta_2"', shift right = 2pt] & \bigoplus_{i, j \in I} \mathcal{F}(U_i \cap U_j)
			\end{tikzcd}
			\]
			is \emph{exact} for all $U \subseteq X$ open, and open covers $\{U_i\}$ of $U$. Here,
			\begin{align*}
				\alpha(s) &= (s|_{U_i})_{i \in I}, \\
				\beta_1((s_i)_{i \in I}) &= (s_i|_{U_i \cap U_j} )_{i, j \in I}, \\
				\beta_2((s_i)_{i \in I}) &= (s_j|_{U_i \cap U_j})_{i, j \in I}.
			\end{align*}
			In category theory, $\alpha$ is the \emph{equalizer}\index{equalizer} of $\beta_1, \beta_2$.

			Exactness means that:
			\begin{itemize}
				\item $\alpha$ is injective (property 1).
				\item $\beta_1 \circ \alpha = \beta_2 \circ \alpha$.
				\item For any $(s_i) \in \bigoplus \mathcal{F}(U_i)$ with $\beta_1((s_i)) = \beta_2((s_i))$, there exists an  $s \in \mathcal{F}(U)$ with $\alpha(s) = (s_i)$ (property 2).
			\end{itemize}
			

			This definition works even if $\mathcal{F}(U)$ is a set, rather than an abelian group.
	\end{itemize}
\end{remark}

\begin{exbox}
	\begin{enumerate}
		\item For $X$ any topological space,
			\[
				\mathcal{F}(U) = \{ \text{continuous functions } f : U \to \mathbb{R}\}
			\]
			is a sheaf.
		\item If $X = \mathbb{C}$ with the Euclidean topology, then
			\[
				\mathcal{F}(U) = \{f : U \to \mathbb{C} \mid f \text{ bounded and holomorphic}\}
			\]
			is not a sheaf, as gluing fails because it does not preserve boundedness.
		\item Let $G$ be a group, and set $\mathcal{F}(U) = G$ for all $U \subseteq X$. Then $\rho_{UV} = \id$. This is a presheaf known as the \emph{constant presheaf}.

			If we give $G$ the discrete topology, set
			\[
				\mathcal{F}(U) = \{f : U \to G \text{ continuous}\}.
			\]
			These are all locally constant functions, and is obviously a sheaf, called the \emph{constant sheaf}.
		\item If $X$ is a variety, denote by $\mathcal{O}_X(U)$ the set of regular functions $f : U \to K$. Then $\mathcal{O}_X(U)$ is a sheaf, called the \emph{structure sheaf}\index{structure sheaf} of $X$.
	\end{enumerate}
\end{exbox}

\begin{definition}
	Let $\mathcal{F}$ be a presheaf in $X$, $p \in X$. Then the \emph{stalk}\index{stalk} of $\mathcal{F}$ at $p$ is
	\[
		\mathcal{F}_p = \{(U, s) \mid U \text{ an open neighbourhood of } p, s \in F(U)\}/ \cong,
	\]
	where $(U, s) \cong (V, t)$ if there exists $W \subseteq U \cap V$, a neighbourhood of $p$, such that
	\[
	s|_W = t|_W.
	\]
	The equivalence class of $(U, s) \in \mathcal{F}_p$ is written as $s_p$, and is the \emph{germ}\index{germ} of $s$ at $p$.
\end{definition}

So the stalk is the set of germs. The stalks should be thought of as the local information of the sheaf around $p$. Note that given a morphism $f : \mathcal{F} \to \mathcal{G}$, we obtain $f_p : \mathcal{F}_p \to \mathcal{G}_p$ by
\[
f_p(U, s) = (U, f_U(s)).
\]

\begin{proposition}
	Let $f : \mathcal{F} \to \mathcal{G}$ be a morphism of sheaves (i.e. a morphism of presheaves). Then $f$ is an isomorphism if and only if $f_p$ is an isomorphism, for all $p \in X$.
\end{proposition}

\begin{proofbox}
	The forward direction is obvious.

	For the other direction, assume $f_p$ is an isomorphism for all $p$. We will show that $f(U) : \mathcal{F}(U) \to \mathcal{G}(U)$ is an isomorphism for all $U$, and then we can define the inverse to $f$ by $(f^{-1})_U = (f_U)^{-1}$.

	First we show $f_U$ is injective. Suppose $s \in \mathcal{F}(U)$ is such that $f_U(s) = 0$. Then for all $p \in U$,
	\[
	f_p((U, s)) =  (U, f_U(s)) = (U, 0) = 0 \in \mathcal{G}_p.
	\]
	Thus $s_p = 0$ since $f_p$ is injective. So there is an open neighbourhood $V_p \subseteq U$ of $p$ such that $s|_{V_p} = 0$.

	But $\{V_p\}$ covers $U$, so by property 1, $s = 0$.

	Now we show $f_U$ is surjective. Let $t \in \mathcal{G}(U)$. Then for all $p \in U$, there exists $s_p \in \mathcal{F}_p$ such that $f_p(s_p) = t_p$, i.e. there exists an open neighbourhood $V_p$ at $p \in U$ and a germ $(V_p, \tilde s_p)$ representing $s_p$ such that
	\[
		(V_p, f_{V_p}(\tilde s_p)) \cong (U, t) = t_p.
	\]
	Shrinking $V_p$ if necessary, we can assume that $f_{V_p}(\tilde s_p) = t|_{V_p}$. Now on $V_p \cap V_q$,
	\[
	f_{V_p \cap V_q} (\tilde s_p |_{V_p \cap V_q} - \tilde s_q|_{V_p \cap V_q} ) = t|_{V_p \cap V_q} - t|_{V_p \cap V_q} = 0.
	\]
	Since we have shown that $f_{V_p \cap V_q}$ is injective, we get
	\[
	\tilde s_p |_{V_p \cap V_q} = \tilde s_q|_{V_p \cap V_q},
	\]
	and so by property 2, there exists $s \in \mathcal{F}(U)$ such that
	\[
	s|_{V_p} = \tilde s_p,
	\]
	for all $p$. Now,
	\[
	f_U(s)|_{V_p} = f_{V_p}(s|_{V_p}) = f_{V_p} (\tilde s_p) = t|_{V_p}.
	\]
	Therefore, $f_U(s) - t = 0$, so by property 1, $f_U(s) = t$. Hence $f_U$ is surjective.
\end{proofbox}

\begin{remark}
	If $f_p : \mathcal{F}_p \to \mathcal{G}_p$ is injective for all $p$, then $f_U : \mathcal{F}(U) \to \mathcal{G}(U)$ is still injective.

	But instead if $f_p : \mathcal{F}_p \to \mathcal{G}_p$ is surjective for all $p$, it need not be the case that $f_U : \mathcal{F}(U) \to \mathcal{G(U)}$ is surjective.
\end{remark}

\subsection{Sheafification}%
\label{sub:sfc}

Given a presheaf $\mathcal{F}$, there exists a sheaf $\mathcal{F}^+$ and a morphism $\theta : \mathcal{F} \to \mathcal{F}^+$, satisfying the following universal property:

For any sheaf $\mathcal{G}$ and morphism $\phi : \mathcal{F} \to \mathcal{G}$, there exists a unique morphism $\phi^+ : \mathcal{F}^+ \to \mathcal{G}$ such that
\[
\begin{tikzcd}
	\mathcal{F} \arrow[r, "\theta"] \arrow[dr, "\phi"] & \mathcal{F}^+ \arrow[d, "\phi^+"] \\
							   & \mathcal{G}
\end{tikzcd}
\]
commutes.

The pair $(\mathcal{F}^+, \theta)$ is unique up to isomorphism. Also $\mathcal{F}_p \cong \mathcal{F}_p^+$ via $\theta_p$, for all $p \in X$.

% lecture 4

The sheafification is defined as follows: define $\mathcal{F}^+(U)$ to be the functions
\[
s : U \to \bigsqcup_{p \in U} \mathcal{F}_p
\]
such that:
\begin{enumerate}[(i)]
	\item $s(p) \in \mathcal{F}_p$ for all $p$,
	\item for each $p \in U$, there exists an open neighbourhood $p \in V \subseteq U$ and an element $t \in \mathcal{F}(V)$ such that
		\[
			(V, t) = s(q),
		\]
		for all $q \in V$.
\end{enumerate}
We define $\theta : \mathcal{F} \to \mathcal{F}^+$ given by
\[
	\mathcal{F}(U) \ni s \mapsto (p \mapsto (U, s) \in \mathcal{F}_p).
\]
We can check that this satisfies the universal property stated previously.

\begin{definition}
	Let $f : \mathcal{F} \to \mathcal{G}$ be a morphism of presheaves. We define:
	\begin{enumerate}
		\item The \emph{presheaf kernel}\index{presheaf kernel} of $f$ as
			\[
				(\ker f)(U) = \ker (f_U : \mathcal{F}(U) \to \mathcal{G}(U)).
			\]
		\item The \emph{presheaf cokernel}\index{presheaf cokernel} of $f$ as
			\[
				(\coker f)(U) = \coker f_U.
			\]
		\item The \emph{presheaf image}\index{presheaf image} as
			\[
				(\img f)(U) = \img f_U.
			\]
	\end{enumerate}
\end{definition}

\begin{remark}
	If $f : \mathcal{F}\to \mathcal{G}$ is a morphism of sheaves, then $\ker f$ is a sheaf. First note that any sub-presheaf of $\mathcal{F}$ satisfies property 1.

	To check property 2, given $s_i \in (\ker f)(U_i) \subseteq \mathcal{F}(U_i)$ for $\{U_i\}$ an open cover of $U$, suppose $s_i |_{U_i \cap U_j} = s_j |_{U_i \cap U_j}$. Then the $s_i$'s glue to given an $s \in \mathcal{F}(U)$. Now,
	\[
	f_U(s) |_{U_i} = f_{U_i} (s_i) = 0,
	\]
	so by property 1, $f_U(s) = 0$. Hence $s \in (\ker f)(U)$.
\end{remark}

\begin{exbox}
	Take $X = \mathbb{P}^1$, or the Riemann sphere, and let $P, Q \in X$ be distinct points.

	Let $\mathcal{G}$ be the sheaf of regular functions on $X$ (or the holomorphic functions on $X$), and let $\mathcal{F}$ be the sheaf of regular functions on $X$ vanishing at $P$ and $Q$ (or the holomorphic functions vanishing at $P$, $Q$).

	Note $\mathcal{F}(U) = \mathcal{G}(U)$ if $U \cap \{P, Q\} = \emptyset$. By Liouville's theorem, $\mathcal{G}(X) = K$, and $\mathcal{F}(X) = 0$.

	Let $U = X \setminus \{P\}$, $V = X \setminus \{Q\}$, and $f : \mathcal{F} \to \mathcal{G}$ the obvious inclusion. Note $\mathcal{G}(U) = K[x]$ by affine geometry, $\mathcal{F}(U) = (x)$. So,
	\begin{align*}
		(\coker f)(X) &= \mathcal{G}(X) / \mathcal{F}(X) = K, \\
		(\coker f)(U \cap V) &= \mathcal{G}(U \cap V) / \mathcal{F}(U \cap V) = 0, \\
		(\coker f)(U) &= \mathcal{G} / \mathcal{F}(U)=  K[x]/(x) = K, \\
		(\coker f)(V) &= K.
	\end{align*}
	Choose $s_U \in (\coker f)(U)$, $s_V \in (\coker f)(V)$. But now $s_U|_{U \cap V} = s_V|_{U \cap V} = 0$, and this would imply that if $\coker f$ were a sheaf, that
	\[
	K \oplus K \subseteq (\coker f)(X).
	\]
\end{exbox}

\begin{remark}
	This shows that $\coker f$ need not be a sheaf. The same is true of $\img f$.
\end{remark}

\begin{definition}
	Let $f : \mathcal{F} \to \mathcal{G}$ be a morphism of sheaves. The \emph{sheaf kernel}\index{sheaf kernel} of $f$ is $\ker f$, the presheaf kernel.

	The \emph{sheaf cokernel}\index{sheaf cokernel} of $f$ is the sheafification of the presheaf cokernel. We still write this as $\coker f$.
\end{definition}

Note that to show $\coker f$ is a presheaf, we need to use the third isomorphism theorem.

We can also show that the sheaf image of $f$ is a subsheaf of $\mathcal{G}$ (prove this).

We say that $f$ is \emph{injective} if $\ker f = 0$, and $f$ is \emph{surjective} if $\img f = \mathcal{G}$. We say that a sequence
\[
\begin{tikzcd}
	\cdots \arrow[r, ""] & \mathcal{F}^{i-1} \arrow[r, "f^i"] & \mathcal{F}^i \arrow[r, "f^{i+1}"] & \mathcal{F}^{i+1} \arrow[r, ""] & \cdots
\end{tikzcd}
\]
is \emph{exact}\index{exact} if $\ker f^{i+1} = \img f^{i}$.

If $\mathcal{F}' \subseteq \mathcal{F}$ is a subsheaf, we write $\mathcal{F} / \mathcal{F}'$ for the sheaf associated to the presheaf
\[
U \mapsto \mathcal{F}(U) / \mathcal{F}'(U).
\]
This is $\coker (\iota : \mathcal{F}' \injto \mathcal{F})$, where $\iota$ is the inclusion.

\begin{lemma}
	Let $f : \mathcal{F} \to \mathcal{G}$ be a morphism of sheaves. Then,
	\begin{align*}
		(\ker f)_p &= \ker (f_p : \mathcal{F}_p \to \mathcal{G}_p), \\
		(\img f)_p &= \img f_p.
	\end{align*}
\end{lemma}

\begin{proofbox}
	We have a map $(\ker f)_p \to \ker f_p$, defined as follows: if $(U, s) \in (\ker f)_p$, then $s \in (\ker f)(U)$, and $(U, s) \in \mathcal{F}_p$. So,
	\[
	f_p (U, s) = (U, f(s)) = (U, 0) = 0 \in \mathcal{G}_p,
	\]
	thus $(U, s) \in \ker f_p$.

	It is easy to see this map is injective: if $(U, s) = 0$ in $\mathcal{F}_p$, then in some neighbourhood $s|_V = 0$. So $(U, s) \sim (V, s|_V) = 0$ in $(\ker f)_p$.

	We now tackle surjectivity. If $(U, s) \in \ker f_p$, then $0 = f_p(U, s) = (U, f_U(s))$ in $\mathcal{G}_p$, so in some neighbourhood $V$, $f_U(s)|_V = 0$.

	Thus $f_V(s|_V) = 0$, so $(U, s) \sim (V, s|_V)$, and $s|_V \in (\ker f)(V)$. Thus $(V, s|_V) \in (\ker f)_p$, which maps to $(U, s) \in \ker f_p$.

	We now prove the appropriate theorem for images. Let $\img' f$ denote the presheaf image. Recall that if $\mathcal{F}$ is a presheaf, then $\mathcal{F}_p \cong \mathcal{F}_p^+$, via $\theta_p$. So it is enough to show there is an isomorphism
	\[
		(\img' f)_p \cong \img f_p.
	\]
% lecture 5

	First, define
	\begin{align*}
		(\img' f)_p &\to \img f_p \\
		(U, s) &\mapsto (U, s),
	\end{align*}
	with $s \in f_U(t)$ for some $t \in \mathcal{F}(U)$, which lives in $\img f_p$ since
	\[
	f_p(U, t) = (U, f_U(t)) = (U, s)
	\]
	First we show injectivity. If $(U, s) = 0$ in $\mathcal{G}_p$, then there exists $p \in V \subseteq U$ such that $s|_V = 0$. But then,
	\[
		(U, s) \sim (V, s|_V) = (V, 0) = 0
	\]
	in $(\img ' f)_p$.

	To show surjectivity, we know that for $(U, s) \in \img f_p$, that there is $(V, t) \in \mathcal{F}_p$ with $f_p(V, t) \sim (U, s)$. We can replace $U$ with the smaller open set $V$, so can assume that $U = V$, and then
	\[
	f_p(U, t) = (U, f_U(t)) \sim (U, s)
	\]
	in $\mathcal{G}_p$. Shrinking $U$ further, we can assume $f_U(t) = s$, and hence
	\[
		(U, s) = (U, f_U(t)) \in (\img ' f)_p.
	\]
\end{proofbox}

\begin{proposition}
	$f : \mathcal{F} \to \mathcal{G}$ is injective if and only if $f_p : \mathcal{F}_p \to \mathcal{G}_p$ is injective for all $p$.

	$f : \mathcal{F} \to \mathcal{G}$ is surjective if and only if $f_p : \mathcal{F}_p \to \mathcal{G}_p$ is surjective for all $p$.
\end{proposition}

\begin{proofbox}
	We have
	\begin{align*}
		f_p \text{ is injective} &\iff \ker f_p = 0 & &\forall p\\
					 &\iff (\ker f)_p = 0 & &\forall p \text{ by the lemma} \\
					 &\iff \ker f = 0.
	\end{align*}
	Note this uses the easy fact that if $\mathcal{F}$ is a sheaf, and $\mathcal{F}_p = 0$ for all $p$, then $\mathcal{F} = 0$. Also,
	\begin{align*}
		f_p \text{ is surjective} &\iff \img f_p = \mathcal{G}_p & &\forall p \\
					  &\iff (\img f)_p = \mathcal{G}_p & & \forall p \text{ by the lemma} \\
					  &\iff \img f = \mathcal{G}.
	\end{align*}
	Hence if $\mathcal{F} \subseteq \mathcal{G}$ are sheaves with $\mathcal{F}_p = \mathcal{G}_p$, we can check that $\mathcal{F} = \mathcal{G}$.
\end{proofbox}

\subsection{Passing between Spaces}%
\label{sub:pbs}

Let $f : X \to Y$ be a continuous map of topological spaces.

Let $\mathcal{F}$ be a sheaf in $X$. Define a sheaf $f_\ast \mathcal{F}$ on $Y$ by
\[
	(f_\ast \mathcal{F})(U) = \mathcal{F}(f^{-1}(U)).
\]
This is the \emph{push-forward}\index{push-forward} of $\mathcal{F}$. This can be checked to be a sheaf.

If $\mathcal{G}$ is a sheaf on $Y$, define the \emph{pull-back}\index{pull-back} sheaf $f^{-1} \mathcal{G}$ to be the sheaf associated to the presheaf
\[
	U \mapsto \{(V, s) \mid V \subseteq f(U), V \text{ open}, s \in \mathcal{G}(V)\} / \sim,
\]
where $(V, s) \sim (V', s')$ if there exists $W$ with $f(U) \subseteq W \subseteq V \cap V'$ with $s|_W = s'|_W$.

\begin{exbox}
	If $f : \{p\} \injto Y$, then $f^{-1} \mathcal{G} = \mathcal{G}_p$, by identifying a sheaf $\mathcal{F}$ on a topological space $X$ with the group $\mathcal{F}(X)$.

	More generally, if $\iota : Z \injto X$ is an inclusion, we often write $\mathcal{F}|_Z$ for the sheaf $\iota^{-1} \mathcal{F}$.

	If $\iota : U \injto X$ is an open subset, then in fact $i^{-1} \mathcal{F} = \mathcal{F}|_Z$ is the sheaf $V \mapsto \mathcal{F}(V)$, for $V \subseteq U$ open.
\end{exbox}

If $s \in \mathcal{F}(U)$, we call $s$ a \emph{section}\index{section} of $\mathcal{F}$ over $U$. We often also write
\[
\mathcal{F}(U) = \Gamma(U, \mathcal{F}),
\]
thinking of $\Gamma(U, \cdot)$ as a covariant functor
\[
\Gamma(U, \cdot) : \mathbf{Presheaves}_X \to \mathbf{Ab}.
\]

\newpage

\section{Affine Schemes}%
\label{sec:as}

Let $A$ be a ring. $\Spec A$ is a topological space analogous to the sheaf of regular functions.

Let $S \subseteq A$ be a multiplicatively closed subset, i.e. $1 \in S$ and whenever $a, b \in S$, we have $a \cdot b \in S$. We define
\[
	S^{-1}A = \{(a, s) \mid a \in A, s \in S\} / \sim,
\]
where $(a, s) \sim (a', s')$ if there exists $s'' \in S$ such that
\[
s''(as' - a's) = 0.
\]
This is called the \emph{localization}\index{localization} of $A$ at $S$.

\begin{exbox}
	\begin{enumerate}
		\item Say $S = \{1, f, f^2, \ldots\}$ for some $f \in A$. Then
			\[
				A_f = S^{-1} A = \left\{ \frac{a}{f^n} \mid a \in A, n \geq 0\right\} / \sim.
			\]
		\item Take $P \subseteq A$ a prime ideal, and $S = A \setminus P$. Then we write $A_P = S^{-1} A$.
	\end{enumerate}
\end{exbox}

Our goal is to now construct the sheaf
\[
\mathcal{O} = \mathcal{O}_{\Spec A}.
\]
For $U \subseteq \Spec A$ open, we write
\begin{align*}
	\mathcal{O}(U) = \biggl\{ s : U \to \bigsqcup_{p \in U} A_p &\mid s(p) \in A_p \wedge \text{for each } p \in U, \exists \, q \in V \subseteq U \\
								    & \text{and }a, f \in A \text{ such that } \forall q \in V, f \not \in q \wedge s(q) = \frac{a}{f} \in A_q \biggr\}
\end{align*}
This is a sheaf, as it is only defined locally.

% lecture 6

The scheme has the following properties. First, for any $p \in \Spec A$, $\mathcal{O}_p = A_p$.

\begin{proofbox}
	We have a map $\mathcal{O}_p \to A_p$ given by $(U, s) \mapsto s(p)$.

	First we show surjectivity. Any element of $A_p$ can be written as $a/f$, for $a \in A$ and $f \not \in p$. We define
	\[
		D(f) = (\Spec A) \setminus V((f)) = \{q \in \Spec A \mid f \not \in q\}.
	\]
	This is an open neighbourhood of $p \in \mathrm{Spec} A$. Then $a/f$ defines a section $s$ of $\mathcal{O}$ over $D(f)$, via
	\[
	s(q) = \frac{a}{f} = A_q.
	\]
	In particular, $s(p) = a/f \in A_p$. So, $(D(f), s) \mapsto a/f$ under this map.

	For injectivity, let $p \in U \subseteq \Spec A$, and $s \in \mathcal{O}(U)$, defining a germ $(U, s) \in \mathcal{O}_p$.

	Suppose $s(p) = 0$. We want to show $(U, s) = 0$ in $\mathcal{O}_p$. We may shrink $U$ and assume there are $a, f \in A$ such that $s(q)=  a/f$ for all $q \in U$. In particular, $f \not \in q$.

	Since $s(p) = 0$, $a/f = 0$ in $A_p$. Thus there exists $h \in A \setminus p$ such that
	\[
		h(a \cdot 1 - 0 \cdot f) = 0,
	\]
	i.e. $h \cdot a = 0$. Let $V = D(f) \cap D(h)$. Then for $q \in V$, $a/f = 0$ in $A_q$.

	Since $h \not \in q$ and $h \cdot a = 0$, $s|_V = 0$, so $(U, s) = 0$. Moreover, $V$ is non-empty as it contains $p$.
\end{proofbox}

Another property is as follows: for any $f \in A$, $\mathcal{O}(D(f)) = A_f$. In particular,
\[
\mathcal{O}(\Spec A) = \mathcal{O}(D(1)) = A_1 = A.
\]

\begin{proofbox}
	Let $\psi : A_f \to \mathcal{O}(D(f))$ be given by
	\[
	\frac{a}{f^n} \mapsto \left( p \mapsto \frac{a}{f^n} \in A_p \right).
	\]
	We need to show this map is injective, and surjective.

	\textbf{Injectivity:} To show $\psi$ is injective, if $\psi(a/f^n) = 0$, then for all $p \in D(f)$, $a/f^n = 0$ in $A_p$. So there exists $h \not \in p$ such that $h \cdot a = 0$, where $h$ may depend on $p$. Let
	\[
		I = \{g \in A \mid g \cdot a = 0\}.
	\]
	So $h \in I$, for all $p$. But also $h \not \in p$, so $I \not \subseteq p$. This is true for all $p \in D(f)$, so
	\[
	V(I) \cap D(f) = \emptyset.
	\]
	Thus,
	\[
	f \in \bigcap_{q \in V(I)} q = \sqrt I,
	\]
	the radical of $I$. Thus $f^m \in I$ for some $n > 0$. Thus $f^m \cdot a = 0$, so in particular $a/f^n = 0$ in $A_f$, by choosing $h = f^m$.

	\textbf{Surjectivity:} is a lot harder. Let $s \in \mathcal{O}(D(f))$. Our goal is to show that $s = \psi(a/f^n)$ for some $a$ and $n$. Let $\{V_i\}$ be an open cover of $D(f)$ such that $s|_{V_i}$ is represented by $a_i/g_i$ with $g_i \not \in p$, for all $p \in V_i$.

	Thus $V_i \subseteq D(g_i)$. By an example sheet question, the open sets of te form $D(h)$ form a basis for the Zariski topology. We can thus assume $V_i = D(h_i)$ for some $h_i \in A$.

	Since $D(h_i) \subseteq D(g_i)$, $V((h_i)) \subseteq V((g_i))$, so $\sqrt{(h_i)} \subseteq \sqrt{(g_i)}$. Therefore $h_i^n \in (g_i)$, for some $n$, so $h_i^n = c_i g_i$ for some $c_i \in A$. Thus,
	\[
	\frac{a_i}{g_i} = \frac{c_i a_i}{h_i^n}.
	\]
	We can thus replace $h_i$ with $h_i^n$, since a prime ideal containing one contains the other, and can now assume that $s|_{V_i}$ is represented by an element of the for $a_i/h_i$, where $V_i = D(h_i)$.

	So far, we have proven that $D(f)$ is covered with open sets $V_i = D(h_i)$, and $s|_{V_i}$ represented by $a_i/h_i$.

	\textbf{Claim:} $D(f)$ can be covered with a finite number of the $D(h_i)$'s, in other words $D(f)$ is \emph{quasicompact}\index{quasicompact}, meaning compact without the Hausdorff condition.

	The proof is as follows:
	\begin{align*}
	&&	D(f) & \subseteq \bigcup_i D(h_i) \\
	&\iff& V((f)) &\supseteq \bigcap V((h_i)) = V \left( \sum_i (h_i) \right) \\
	&\iff& \sqrt{(f)} & \subseteq \sqrt{ \sum_i (h_i) } \\
	&\iff& f &\in \sqrt{\sum_i (h_i)} \\
	&\iff& f^n &\in \sum_I (h_i)
	\end{align*}
	for some $n$. Thus,
	\[
	f^n = \sum b_i h_i
	\]
	for some finite set of $h_i$'s. Thus, by running the above proof backwards, we find that the finite number of these $D(h_i)$ cover $D(f)$.

	Now we have a finite cover of $D(f)$. On $D(h_i) \cap D(h_j)$, $a_i/h_i$ and $a_j/h_j$ both represent $s|_{D(h_i) \cap D(h_j)}$. But,
	\[
	\psi : A_{h_i h_j} \to \mathcal{O}(D(h_i, h_j)) = \mathcal{O}(D(h_i) \cap D(h_j))
	\]
	is injective, by proof of injectivity. So, $a_i/h_i = a_j/h_j$ in $A_{h_i h_j}$. Thus for some $n$,
	\[
		(h_i h_j)^n (h_j a_i - h_i a_j) = 0.
	\]
	We can choose $n$ to work for all $i, j$. Rewriting this as
	\[
	h_j^{n+1}(h_i^n a_i) - h_i^{n+1}(h_j^n a_j) = 0,
	\]
	and replacing each $h_i$ with $h_i^{n+1}$ and $a_i$ with $h_i^n a_i$, the ratio is the same, and so $s$ is still represented by $a_i/h_i$, but also $h_j a_i = h_i a_j$ for all $i, j$. Let
	\[
	a = \sum b_i a_i,
	\]
	where we recall that
	\[
	f^n = \sum b_i h_i.
	\]
	Then for any $j$,
	\[
	h_j a = \sum h_j b_i a_i = \sum_i h_i b_i a_j = f^n a_j,
	\]
	so $a_j/h_j = a/f^n$, for all $j$. So $\psi(a/f^n) = s$.
\end{proofbox}

% lecture 7

\begin{definition}
	A \emph{ringed space}\index{ringed space} is a pair $(X, \mathcal{O}_X)$ where:
	\begin{itemize}
		\item $X$ is a topological space,
		\item $\mathcal{O}_X$ is a sheaf of rings.
	\end{itemize}

	A \emph{morphism}\index{morphism of ringed spaces} of ringed spaces $f : (X, \mathcal{O}_X) \to (Y, \mathcal{O}_Y)$ is data such that:
	\begin{itemize}
		\item $f : X \to Y$ is continuous.
		\item $f^{\#} : \mathcal{O}_Y \to f_\ast \mathcal{O}_X$ is a morphism of sheaves of rings, i.e.
			\[
			f_U^{\#} : \mathcal{O}_Y(U) \to (f_\ast \mathcal{O}_X)(U) = \mathcal{O}_X(f^{-1}(U))
			\]
			is a ring homomorphism.
	\end{itemize}
\end{definition}

\begin{exbox}
	\begin{enumerate}
		\item Let $X$ be a variety, and $\mathcal{O}_X$ the sheaf of regular functions on $X$, so
			\[
				\mathcal{O}_X(U) = \{f : U \to K \text{ such that $f$ is regular}\}.
			\]
			A morphism $f : X \to Y$ of varieties is a continuous map $f : X \to Y$ inducing a map, for all $U \subseteq Y$, by $\mathcal{O}_Y(U) \to \mathcal{O}_X(f^{-1}(U))$, where $\varphi \mapsto \varphi \circ f$.
		\item Let $X$ be a topological space, then we can consider
			\[
				\mathcal{O}_X(U) = \{f : U \to \mathbb{R} \mid f \text{ continuous}\}.
			\]
		\item Let $X$ be a $C^\infty$-manifold, and
			\[
				\mathcal{O}_X(U) = \{f : U \to \mathbb{R} \mid f C^\infty\}.
			\]
			Then $f : X \to Y$ a continuous map between $C^\infty$ manifolds is $C^\infty$ if and only if, for any $C^\infty$ function $\varphi : U \to \mathbb{R}$, and $U \subseteq Y$, $f \circ \varphi : f^{-1}(U) \to \mathbb{R}$ is $C^\infty$.
	\end{enumerate}	
\end{exbox}

\begin{remark}
	All of these example have the feature that $\mathcal{O}_{X, p}$, for $p \in X$, is a local ring, i.e. one with a unique maximal ideal.

	Indeed, take $m_p = \{(U, f) \mid f(p) = 0\}$. Note $(U, f) \sim (V, f|_V)$ for some $V$ with $f|_V$ non-vanishing if $f(p) \neq 0$.

	So every element in $\mathcal{O}_{X, p} \setminus m_p$ is invertible, hence $m_p$ is the unique maximal ideal.
\end{remark}

Note that given a morphism of ringed space $f : (X, \mathcal{O}_X) \to (Y, \mathcal{O}_Y)$ in the previous three examples, we get a map
\[
f_p^{\#} : \mathcal{O}_{Y, f(p)} \to \mathcal{O}_{X, p}
\]
for $p \in X$ defined by
\[
f_p^{\#}(U, \varphi) = (f^{-1}(U), \varphi \circ f).
\]
Note that $(U, \varphi) \in m_{f(p)}$ if and only if $(f^{-1}(U), \varphi \circ f) \in m_p$.

In particular, $(f_p^{\#})^{-1}(m_p) = m_{f(p)}$.

\begin{definition}
	A \emph{locally ringed space}\index{locally ringed space} is a ringed space $(X, \mathcal{O}_X)$ such that $\mathcal{O}_{X, p}$ is a local ring for all $p \in X$.

	A \emph{morphism}\index{morphism of locally ringed spaces} $f : (X, \mathcal{O}_X) \to (Y, \mathcal{O}_Y)$ of locally ringed spaces is a morphism of ringed spaces such that the induced morphism
	\[
	f_p^{\#} : \mathcal{O}_{Y, f(p)} \to \mathcal{O}_{X, p}
	\]
	is a local homomorphism.

	Here $f_p^{\#}$ is defined by
	\[
	\mathcal{O}_{Y, f(p)} \ni (U, \varphi) \mapsto (f^{-1}(U), f_U^{\#}(\varphi)) \in \mathcal{O}_{X, p}.
	\]
\end{definition}

A homomorphism  $g : A \to B$ of local rings is a ring homomorphism such that $g^{-1}(m_B) = m_A$. Note that $g(A \setminus m_a) = g(A^\ast) \subseteq B^\ast = B \setminus m_B$, so we always have $g^{-1}(m_B) \subseteq m_A$.

The motivating example is $(\Spec A, \mathcal{O}_{\Spec A})$, which we will show is a locally ringed space. Recall that $\mathcal{O}_{\Spec A, p} = A_p$, which has a unique maximal ideal
\[
m = \left\{ \frac{a}{s} \biggm| a \in p, s \not \in p\right\}.
\]
There is a natural map $A \to A_p$ by $a \mapsto a/1$, and $m \subseteq A_p$ is the ideal generated by the image of $p$ under this map. This is often written as $p A_p$.

We call $(\Spec A, \mathcal{O}_{\Spec A})$ an \emph{affine scheme}\index{affine scheme}.

\begin{theorem}
	The category of affine schemes with morphisms being morphisms of locally ringed spaces is equivalent to the opposite category of rings.
\end{theorem}

\begin{proofbox}
	We need to show that:
	\begin{itemize}
		\item If $\varphi : A \to B$ is a homomorphism of rings, we get a morphism of locally ringed spaces
			\[
				(f, f^{\#}) : (\Spec B, \mathcal{O}_{\Spec B}) \to (\Spec A, \mathcal{O}_{\Spec A}).
			\]
		\item Any morphism of locally ringed spaces is induced b a homomorphism $\varphi : A \to B$ of rings.
	\end{itemize}
	For the first part, we define $f : \Spec B \to Spec A$ by
	\[
	f(p) = \varphi^{-1}(p).
	\]
	Note that $\varphi^{-1}(p)$ is prime, as if $a \cdot b \in \varphi^{-1}(p)$, then $\varphi(a) \varphi(b) = \varphi(ab) \in p$, so say $\varphi(a) \in p$. Then $a \in \varphi^{-1}(p)$.

	Now we show $f$ is continuous. Note for any $I \subseteq A$,
	\begin{align*}
		f^{-1}(V(I)) &= \{p \in \Spec B \mid f(p) \supseteq I\} = \{p \in \Spec B \mid \varphi^{-1}(p) \supseteq I\} \\
			     &= \{p \in \Spec B \mid p \supseteq \varphi(I)\} \\
			     &= V(\varphi(I)).
	\end{align*}
	
\end{proofbox}


% lecture 8

\newpage

\printindex

\end{document}
