\documentclass[12pt]{article}

\usepackage{ishn}

\makeindex[intoc]

\begin{document}

\hypersetup{pageanchor=false}
\begin{titlepage}
	\begin{center}
		\vspace*{1em}
		\Huge
		\textbf{III Symmetries, Particles and Fields}

		\vspace{1em}
		\large
		Ishan Nath, Michaelmas 2024

		\vspace{1.5em}

		\Large

		Based on Lectures by Prof. Matthew Wingate

		\vspace{1em}

		\large
		\today
	\end{center}
	
\end{titlepage}
\hypersetup{pageanchor=true}

\tableofcontents

\newpage

% lecture 1

\section{Introduction to Symmetries}%
\label{sec:intro_sym}

Recall Newton's second law:
\[
m \frac{\diff^2 \mathbf{x}}{\diff t^2} = \mathbf{F}(\mathbf{x}).
\]
This simplifies if we know $F$ is rotationally symmetric, i.e. $\mathbf{F}(\mathbf{x}) = F(r) \mathbf{\hat r}$. Then $\mathbf{L} = \mathbf{x} \times \mathbf{p}$ is conserved, and trajectories lie in planes containing the origin.

Now consider Lagrangian mechanics, with Lagrangian $L(q_i, \dot q_i, t)$. The principle of least action says
\[
S = \int_{t_1}^{t_2} \diff t \, L(q_i(t), \dot q_i(t), t)
\]
is minimized by classical trajectories. Hence Euler-Lagrange gives
\[
\frac{\partial L}{\partial q_i} - \frac{\diff}{\diff t} \left( \frac{\partial L}{\partial \dot q_i} \right) = 0.
\]
N\"oether's theorem says that invariance of $L$ under some coordinate transform corresponds to an associated conserved quantity.

\begin{exbox}
	Consider a particle in three dimension, with a potential:
	\[
	L = \frac{1}{2} m (\dot x^2 + \dot y^2 + \dot z^2) - U(x, y, z),
	\]
	which is independent of $t$, hence invariant under $t \mapsto t + \delta t$. This implies that the Hamiltonian $H = T + U$ is conserved. If we transform into canonical momenta $p_i = \frac{\partial L}{\partial \dot x_i} = m \dot x_i$, then
	\[
	H(x_i, p_i, t) = \sum \dot x_i p_i - L = \sum \dot x_i \frac{\partial L}{\partial \dot x_i} - L
	\]
	is invariant by Euler-Lagrange:
	\[
	\frac{\diff H}{\diff t} = \sum \ddot x_i \frac{\partial L}{\partial \dot x_i} - \sum x_i \frac{\diff}{\diff t} \left( \frac{\partial L}{ \partial \dot x_i} \right) - \dot x_i \frac{\partial L}{\partial x_i} - \ddot x_i \frac{\partial L}{\partial \dot x_i} - \frac{\partial L}{\partial t} = 0.
	\]

	If $L$ is invariant under $x \mapsto x + \delta x$, then
	\[
		\frac{\partial L}{\partial x} = 0 \implies \frac{\partial L}{\partial \dot x} = p_x = \text{constant}.
	\]
	If $L$ is invariant under rotations about the $z$-axis, then the $z$-component of angular momentum, $xp_y - yp_x$, is constant. The best way to see is transform $L$ into cylindrical coordinates:
	\[
	L = \frac{1}{2} m (\dot \rho^2 - \rho^2 \dot \theta + \dot z^2) - U(\rho, z).
	\]
	So the invariance under rotations means
	\[
	\frac{\partial L}{\partial \theta} = 0 \implies \frac{\partial L}{\partial \dot \theta} = 0 = m \rho^2 \dot \theta = xp_y - yp_x
	\]
	is constant.
\end{exbox}

\subsection{Symmetries in Quantum Mechanics}%
\label{sub:sym_qm}

Given a system whose states are element of a Hilbert space $\mathcal{H}$, a symmetry means there exists some invertible operator $U$ acting on $\mathcal{H}$ which preserves inner products, up to an overall phase $e^{i\phi}$.

\begin{definition}
	Let $\ket \psi, \ket \phi$ be any normalized vectors in $\mathcal{H}$. Denote $\ket{U \psi} = U \ket \psi$, and $\ket{U \phi} = U \ket \phi$.

	$U$ is a \emph{symmetry transformation}\index{symmetry transformation} if
	\[
		|\braket{U\phi | U \psi}| = |\braket{\phi|\psi}|.
	\]
\end{definition}

\begin{proposition}[Wigner's Theorem]
	Symmetry transformation operators are either linear and unitary, or antilinear and antiunitary.
\end{proposition}

Antilinear and antiunitary means
\[
U(a\ket \psi + \beta \ket \phi) = a^\ast U \ket \psi + b^\ast U \ket \phi,
\]
\[
	\braket{U\phi | U\psi} = \braket{\phi|\psi}^\ast.
\]

Suppose we have a system with a time-independent Hamiltonian. Then we can write down
\[
	\braket{\psi(t)}=  e^{-iHt} \ket{\psi(0)},
\]
by Schr\"odinger's equation with $\hbar = 1$. In the first case, note
\begin{align*}
	\braket{U \phi | U \psi(t)} &= \braket{\phi|\psi(t)} \\
				    &= \braket{\phi | e^{-iHt}| \psi(0)}.
\end{align*}

We should find the same result by transforming $\ket{\psi(0)}$ before time evolution:
\begin{align*}
	\braket{U\phi | U \psi(t)} &= \braket{U\phi | e^{-iHt} | U\psi(0)} \\
				   &= \braket{\phi | U^{\dagger} e^{-iHt} U | \psi(0)}.
\end{align*}
Equating these, we find
\[
	U^{\dagger} e^{-iHt}U = e^{-iHt} \implies [U, H] = 0.
\]

\begin{exbox}
	If $H$ commutes with $\mathbf{p}$, then $H$ cannot depend on $\mathbf{x}$, as
	\[
		[x_i, p_j] = i \delta_{ij} \neq 0
	\]
	generally. So $H$ is invariant under translation $\mathbf{x} \mapsto \mathbf{x} + \mathbf{a}$, and this is generated by unitary operators $\exp(i \mathbf{p} \cdot \mathbf{a})$.

	If $H$ is rotationally symmetric, then any momentum operator $\mathbf{J}$ or $\mathbf{L}$ commutes with $H$.
\end{exbox}

\newpage

\section{Lie Groups and Algebras}%
\label{sec:lie_g_alg}

\subsection{Lie Groups}%
\label{sub:lie_g}

Recall the definition of a group: a set together with a relation which has an identity, inverses and is associative.

Also recall a group is abelian if $g \cdot h = h \cdot g$ for all $g, h \in G$.

% lecture 2

\begin{definition}
	A \emph{manifold}\index{manifold} is a space which looks Euclidean, like $\mathbb{R}^n$, on small scales, in small neighbourhoods.

	A \emph{differentiable manifold}\index{differentiable manifold} is one which satisfies certain smoothness conditions.
\end{definition}

\begin{definition}
	A \emph{Lie group}\index{Lie group} consists of a differentiable manifold $G$ along with a binary operation $\cdot$, such that the group axioms hold, and that $\cdot$ and inverse are smooth operations.
\end{definition}

\subsection{Matrix Lie Groups}%
\label{sub:mlg}

For example, the general linear group $\mathsf{GL}(n, \mathbb{F})$ is the group of invertible $n \times n$ matrices over a field $\mathbb{F}$. So,
\[
	\mathsf{GL}(n, \mathbb{F}) = \{ M \in \mathrm{Mat}_n(\mathbb{F}) \mid \det M \neq 0\}.
\]
The group operation is simply matrix multiplication.

The dimension of $\mathsf{GL}(n, \mathbb{R})$ is $n^2$, as there are $n^2$ free parameters. For $\mathsf{GL}(n, \mathbb{C})$, we have real dimension $2n^2$, and complex dimension $n^2$.

Important subgroups of $\mathsf{GL}(n, \mathbb{F})$ are:
\begin{itemize}
	\item The special linear group
		\[
			\mathsf{SL}(n, \mathbb{F}) = \{M \in \mathsf{GL}(n, \mathbb{F}) \mid \det M = 1\}.
		\]
	\item $\mathsf{SL}(n, \mathbb{R})$ has dimension $n^2 - 1$.
	\item The orthogonal group
		\[
			\mathsf O(n) = \{M \in \mathsf{GL}(n, \mathbb{R}) \mid M^T M = I\}.
		\]
		This implies $\det M = \pm 1$. We can also define
		\[
			\mathsf{SO}(n) = \{M \in \mathsf O(n) \mid \det M = 1\}.
		\]
	\item Pseudo-orthogonal group. Define an $(n + m) \times (n + m)$ matrix by
		\[
		\eta =
		\begin{pmatrix}
			I_n & 0 \\
			0 & -I_M
		\end{pmatrix}.
		\]
		Then we can define
		\[
			\mathsf{O}(n, m) = \{M \in \mathsf{GL}(n + m, \mathbb{R}) \mid M^T \eta M = \eta\}.
		\]
		Note $M \in \mathsf{SO}(n, m) \iff \det M = 1$.
	\item Unitary.
		\[
			\mathsf{U}(n) = \{ M \in \mathsf{GL}(n, \mathbb{C}) \mid M^{\dagger} M = I\}.
		\]
		Similarly have $\mathsf{SU}(n)$.
	\item Pseudounitary.
		\[
			\mathsf{U}(n, m) = \{M \in \mathsf{GL}(n, \mathbb{C}) \mid M^{\dagger} \eta M = \eta\}.
		\]
	\item Symplectic group. Define a fixed, antisymmetric $2n \times 2n$ matrix, e.g.
		\[
		\Omega =
		\begin{pmatrix}
			0 & I_n \\
			-I_n & 0
		\end{pmatrix}.
		\]
		Then,
		\[
			\mathsf{Sp}(2n, \mathbb{R}) = \{M \in \mathsf{GL}(2n, \mathbb{R}) \mid M^{T} \Omega M = \Omega\}.
		\]
		We can show that $\det M = 1$ using the Pfaffian.
\end{itemize}

\begin{definition}
	Given a $2n \times 2n$ antisymmetric matrix $A$, its \emph{Pfaffian}\index{Pfaffian} is given by
	\[
	\mathrm{Pf} A = \frac{1}{2^n n!} \eps_{i_1 i_2 \cdots i_{2n}} A_{i_1 i_2} A_{i_3 i_4} \cdots A_{i_{2n-1} i_{2n}}.
	\]
\end{definition}

\subsection{Group Elements as Transformations}%
\label{sub:ge_t}

We can define actions of group elements $g \in G$ on a set $X$.

\begin{definition}
	The \emph{left action}\index{left action} of $G$ on $X$ is a map $L : G \times X \to X$ such that $L(e, x) = x$, and
	\[
	L(g_2, L(g_1, x)) = L(g_2 g_1, x),
	\]
	for all $x \in X$ and $g_1, g_2 \in G$. In more usual notation, for all $g \in G$, we can associate a map $g : X \to X$ as $g(x) = gx$.
\end{definition}

\begin{definition}
	The \emph{right action}\index{right action} of $G$ on $X$ is defined by $g : X \to X$ such that $g(x) = xg^{-1}$, for all $x \in X$, $g \in G$. The inverse preserves under composition, so
	\[
	g_2(g_1(x)) = x g_1^{-1} g_2^{-1} = (g_2 g_1)(x).
	\]
\end{definition}

\begin{definition}
	The action of \emph{conjugation}\index{conjugation} by $G$ on $X$ is the action defined by
	\[
	g(x) = g x g^{-1},
	\]
	for $g \in G$, $x \in X$.
\end{definition}

\begin{definition}
	Given a group $G$ and a set $X$, an \emph{orbit}\index{orbit} of an element $x \in X$ is the set of elements of $X$ in the image of $G$.
\end{definition}

\begin{exbox}
	If the action is a left action, then the orbit of $x \in X$ is
	\[
		Gx = \{gx \mid g \in G\}.
	\]
\end{exbox}

It can be shown that the set of orbits under $G$ partition $X$.

In $\mathbb{R}^n$, orthogonal matrices $\mathsf{O}(n)$ represent rotations and reflections, and preserve the inner product; similarly for $\mathsf{U}(n)$.

We can parametrize $\mathsf{SO}(2)$ as
\[
\mathsf{SO}(2) = \left\{ R(\theta) =
	\begin{pmatrix}
		\cos \theta & - \sin \theta \\
		\sin \theta & \cos \theta
	\end{pmatrix}
	 \biggm|  \theta \in [0, 2\pi]\right\}.
\]
$\cos$, $\sin$ are smooth. We can show that $R(\theta_2)R(\theta_1) = R(\theta_1 + \theta_2)$.

$\mathsf{SO}(3)$ gives rotations of vectors in $\mathbb{R}^3$. The axis of rotation is given by a unit vector $\mathbf{n} \in S^2$, and we also have an angle $\theta$.

Note that rotation by $\theta \in [-\pi, 0]$ about $\mathbf{n}$ is equivalent to rotation by $-\theta$ about $- \mathbf{n}$, so we can confine $\theta \in [0, \pi]$.

Hence we can depict the manifold of $\mathsf{SO}(3)$ as a ball of radius $\pi$ in $\mathbb{R}^3$, where antipodal points are identified: $\pi \mathbf{n} = - \pi \mathbf{n}$.

% lecture 3

The \emph{pseudo-orthogonal group}\index{pseudo-orthogonal group} $\mathsf{SO}(n, m)$ act on vectors in $\mathbb{R}^{n + m}$, and preserve the scalar product $v_2^T \eta v_1$ for $v_1, v_2 \in \mathbb{R}^{n+m}$.

For example,
\[
\mathsf{SO}(1, 1) = \left\{
	\begin{pmatrix}
		\cosh \psi & \sinh \psi \\
		\sinh \psi & \cosh \psi
	\end{pmatrix}
\biggm| \psi \in \mathbb{R} \right\}.
\]
$\mathsf{SO}(1, 1)$ is an example of a non-compact group.

\subsection{Parametrization of Lie Groups}%
\label{sub:param_lie}

At least in small neighbourhoods, we can assign coordinates
\[
x = (x^1, \ldots, x^n) \in \mathbb{R}^n,
\]
such that $g(x) \in G$. Closure says that $g(y) g(x) = g(z)$, and smoothness says that the components of $z$ are continuously differentiable functions of $x$ and $y$, so
\[
z^n = \phi^n(x, y).
\]
We can choose the coordinates at the origin such that $g(0) = e$. Then $g(0) g(x) = g(x)$, so
\[
\phi^r(x, 0) = x^r, \qquad \phi^r(0, y) = y^r.
\]
From inverses, for each $x$ there exists $\bar x$ such that $g(\bar x) = g(x)^{-1}$, hence
\[
\phi^r(\bar x, x) = 0 = \phi^r(x, \bar x).
\]
Finally, associativity means $g(z)[g(y)g(x)] = [g(z)g(y)]g(x)$, hence
\[
\phi^r(\phi(x, y), z) = \phi^r(x, \phi(y, z)).
\]
\subsection{Lie Algebras}%
\label{sub:lie_a}

Lie groups are hard to quantify. Instead, we look at lie algebras, which are a linearization of the lie group.

A lie group is homogeneous: any neighbourhood can be mapped to any other neighbourhood. We will linearize near the identity of $G$.

\begin{definition}
	A \emph{Lie algebra}\index{Lie algebra} is a vector space $V$, which additionally has a vector product, the \emph{Lie bracket}\index{Lie bracket} $[\cdot, \cdot] : V \times V \to V$ possessing the following properties: for $X, Y, Z \in V$,
	\begin{enumerate}
		\item antisymmetry: $[X, Y] = -[Y, X]$.
		\item Jacobi identity: $[X, [Y, Z]] + [Y, [X, Z]] + [Z, [X, Y]] = 0$.
		\item linearity: for $\alpha, \beta \in \mathbb{F}$, $[\alpha X + \beta Y, Z] = \alpha[X, Z] + \beta[Y, Z]$.
	\end{enumerate}
\end{definition}

\begin{remark}
	Any vector space which has a vector product $\ast$ can be made into a Lie algebra with Lie bracket
	\[
		[X, Y] = X\ast Y - Y\ast X.
	\]
\end{remark}

Given a Lie algebra $V$, choose a basis $\{T_a\}$. The basis vectors are called the \emph{generators}\index{generators} of the Lie algebra.

Write their Lie brackets as
\[
	[T_a, T_b] = f\indices{^{c}_{ab}} T_c,
\]
with $f\indices{^{c}_{ab}} \in \mathbb{F}$ called the \emph{structure constants}\index{structure constants}. The properties imply:
\begin{itemize}
	\item antisymmetry $\implies f\indices{^{c}_{ba}} = - f\indices{^{c}_{ab}}$.
	\item Jacobi $\implies f\indices{^{e}_{ad}} f\indices{^{d}_{bc}} + f\indices{^{e}_{cd}} f\indices{^{d}_{ab}} + f\indices{^{e}_{bd}} f\indices{^{d}_{ca}} = 0$.
\end{itemize}

General elements of Lie algebras are linear combinations of $\{T_a\}$. So $X \in V$ can be written as $X^a T_a$, where $X^a \in \mathbb{F}$, and
\[
	[X, Y] = X^a Y^b f\indices{^{c}_{ab}} T_c.
\]

\subsection{Lie Groups and their Lie Algebras}%
\label{sub:lg_la}

We start with $\mathsf{SO}(2)$, where
\[
g(\theta) =
\begin{pmatrix}
	\cos \theta & -\sin \theta \\
	\sin \theta & \cos \theta
\end{pmatrix}.
\]
The identity is $e = I_2 = g(0)$. Near the identity, $\theta$ is small, and
\[
\sin \theta = \theta - \frac{\theta^3}{3} + \cdots, \qquad \cos \theta = 1 - \frac{\theta^2}{2} + \cdots
\]
Hence,
\[
g(\theta) = I_2 + \theta
\begin{pmatrix}
	0 & -1 \\ 1 & 0
\end{pmatrix} - \frac{\theta^2}{2} I_2 + \mathcal{O}(\theta^3) = e + \theta \frac{\diff g}{\diff \theta} \biggr|_{0} + \mathcal{O}(\theta^2).
\]
The linear term is the ``tangent'' to the manifold. We have a one-dimensional tangent space at $e$, and we claim that this is the Lie algebra of $\mathsf{SO}(2)$, i.e.
\[
L(\mathsf{SO}(2)) = T_e(\mathsf{SO}(2)) = \left\{
	\begin{pmatrix}
		0 & -a & a & 0
	\end{pmatrix}
\biggm| a \in \mathbb{R} \right\}.
\]

For $\mathsf{SO}(n)$, we can show the dimension is $\frac{n(n-1)}{2} = d$. Choose coordinates $x_1, \ldots, x_d$, and consider a single-parameter family of $\mathsf{SO}(n)$ elements
\[
M(t) = M(x(t)) \in \mathsf{SO}(n),
\]
such that $M(0) = I_n$. Then,
\[
0 = \frac{\diff}{\diff t} (M^T(t) M(t)) = \frac{\diff M^T}{\diff t} M + M^T \frac{\diff M}{\diff t}.
\]
Looking at $t = 0$, we find
\[
\frac{\diff M^T}{\diff t} = - \frac{\diff M}{\diff t},
\]
hence matrices in the tangent space are anti-symmetric. Moreover they are also traceless.

% lecture 4

For unitary groups, we again let $M(t)$ be a curve in $\mathsf{SU}(n)$ with $M(0) = I$. For small $t$, write
\[
M(t) = I + t X + \mathcal{O}(t^2).
\]
From unitarity, $I = M^{\dagger} M$, so looking at the expansion,
\[
I = I + t(X + X^\dagger) + \mathcal{O}(t^2),
\]
hence $X^\dagger = -X$, is anti-hermitian. We also claim $X$ is traceless for $\mathsf{SU}(n)$. Indeed, looking at $\det M$, its expansion is
\[
1 = \det M = 1 + t \Tr(X) + \mathcal{O}(t^2).
\]
\subsection{Lie Algebras of a Matrix Lie Group}%
\label{sub:la_mlg}

Consider two curves through the identity $e$ of some Lie group, $g_1(x(t))$ and $g_2(y(t))$, with $X_1 = \dot g_1|_0$, $X_2 = \dot g_2|_0$. The product is
\[
g_3(z(t)) = g_2(y(t)) g_1(x(t)) \in G.
\]
Then,
\[
\dot g_3|_0 = (\dot g_1 g_2 + g_1 \dot g_2)|_0 = X_1 + X_2 \in T_e(G).
\]
The Lie bracket arises from the group commutator.

\begin{definition}
	The \emph{group commutator}\index{group commutator} of $g_1, g_2 \in G$ is
	\[
		[g_1, g_2] = g_1^{-1} g_2^{-1} g_1 g_2 \in G.
	\]
\end{definition}

Let $g_1(t), g_2(t)$ be two curves through the identity, and
\[
g_i(t) = c + t X_i + t^2 W_i + \mathcal{O}(t^3).
\]
Then,
\begin{align*}
	g_1(t) g_2(t) &= e + t(X_1 + X_2) + t^2(X_1 X_2 + W_1 + W_2) + \mathcal{O}(t^3), \\
	g_2(t) g_1(t) &= e + e(X_1 + X_2) + t^2(X_2 X_1 + W_1 + W_2) + \mathcal{O}(t^3).
\end{align*}
Therefore,
\[
	h(t) = [g_2(t)g_1(t)]^{-1} g_1(t) g_2(t) = e + t^2[X_1, X_2] + \cdots.
\]
So if $h(t) \in G$, then the tangent to $h(t)$ at $e$ is $[X_1, X_2] \in L(G)$.

Now we can think of tangent spaces to $G \subseteq \mathsf{GL}(n, \mathbb{F})$ at a general element $p$, $T_p(G)$.

Let $g(t)$ be a curve in the manifold through $p$ with $g(t_0) = p$, so
\[
g(t_0 + \eps) = g(t_0) + \eps \dot g (t_0) + \mathcal{O}(\eps^2).
\]
Both $g(t_0)$ and $g(t_0 + \eps)$ are in $G$, so there exists $h_p(\eps) \in G$ such that
\[
g(t_0 + \eps) = g(t_0) h_p(\eps),
\]
where $h_p(0) = e$. For small $\eps$,
\[
h_p(\eps) = e + \eps X_p + \mathcal{O}(\eps^2)
\]
for some $X_p \in L(G) = T_e(G)$. Neglecting higher order terms,
\begin{align*}
	e + \eps X_p &= h_p(\eps) = g(h_0)^{-1} g(t_0 + \eps) \\
		     &= g(t_0)^{-1} [g(t_0) + \eps \dot g (t_0)] \\
		     &= e + \eps g(t_0)^{-1} \dot g(t_0),
\end{align*}
where $g(t)^{-1} \dot g(t) = X_p \in L(G)$.

Conversely, for any $X \in L(G)$, there exists a unique curve $g(t)$ with $g^{-1}(t) \dot g(t) = X$, and $g(0) = g_0$. This is a consequence of the existence and uniqueness of solutions to ODEs. The solution of this ODE is
\[
g(t) = g_0 \exp tX,
\]
where
\[
\exp tX = \sum_{k = 0}^\infty \frac{1}{k!} (tX)^k.
\]
Given an $X \in L(G)$, the curve
\[
g_X(t) = \exp tX,
\]
which forms an abelian subgroup of $G$ generated by $X$. Note $g_x(t)$ is isomorphic to $(\mathbb{R}, +)$ if only $g_x(0)=  e$, and $S^1$ if $g_x(t_0) = e$ for some $t_0 \neq 0$.

% lecture 5

\newpage

\printindex

\end{document}
